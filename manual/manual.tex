\documentstyle{article}

\renewcommand{\thesection}{\Alph{section}.}

\def\t{\texttt}
\def\e{\emph}
\def\mr{\mathrm}

\begin{document}
\title{\Huge\bf ATOM\\manual}
\author{Vito D.P. Servedio}
\date{\today}
\maketitle

\section{Description}
ATOM is a program which allows to calculate the ground state electron
density and energy of isolated atoms and ions.
It is based on the Spin Density Functional Theory (SDFT) approach to the
many body problem. 
In most of the cases the Local Spin Density Approximation (LSDA) is
used, the only exception being the choice of a source free exchange
and correlation (XC) field.
The basic equations underlying the solution of the atomic problem are
chosen to be the Kohn-Sham-Dirac ones.
As a result, the effective one electron eigenfunctions are represented
by bispinors. 
This treatment is commonly referred to as ``fully relativistic''.
The reader is invited to refer to the present and wide literature on
the subject.

The program ATOM was written in \e{C} without the use of any machine
dependent code and can thus be compiled on any architecture, although
only \e{HP-UX} and \e{Linux} systems were personally tested.
The matrix diagonalization routine is written in \e{Fortran77} and was
imported from the \e{EISPACK} routines. 
In absence of a fortran compiler, a standard \e{f2c} conversion of
this routine can be done before the compilation process.

\section{Running the program}
In the following we refer to the executable name as \t{ATOM}, but
you may have changed it according to your personal taste.
Of course the directory containing the executable must be present in
your PATH environmental variable. 
To run the program just type its name on the command line.
The program requires some input files to run. 
In case these files are not present, they will be created
automatically.
These files, the structure of which will be dealt with later, are:
\begin{itemize}
\item \t{relat.ini}, containing the list of basic options and
parameters. It is a mandatory file.
\item \t{El\_structure}, containing the electronic configuration of
the atom. This file may be substituted by the specification on the
command line of the atom or ion symbol to be calculated.
\item \t{niocc}, which is required in the case of a calculation with
non integer occupation numbers.
\end{itemize}
\section{Command line parameters}
In the following we shall describe the accepted command line
parameters:
\begin{description}
\item[-f filename] The program will use the initialization file
\e{filename} instead of the default \t{relat.ini}.
\item[-pN] the program will output the density of the orbital number N
to a file.
\item[-M] It will output the present manual in the file \t{MANUAL.ps}
and exit.
\item[-h] A short list of these command line options will be printed
on the standard output device and exit.
\item[:elem] The ground state electronic configuration of the atom or
ion \e{elem} will be used instead of reading the electronic
configuration \t{El\_structure} file.
\end{description}
As an example to run the program for the potassium plus one ion with
the default settings, you could type twice:\\
\t{ATOM :K+}

\section{Ini file}
\begin{description}

% Master flags
\item[NRMAX] 
	Number of points in the radial mesh. The memory will be
	dynamically allocated. 
\item[MESH]
	Kind of radial mesh \{$i=0\ldots(\mr{NRMAX-1})$\}:
	\begin{description}
	\item[0]
		Linear grid.
		\( r_i=\frac{R_{\mr{max}}}{\mr{NRMAX}} (i+1) \)
	\item[1]
		Quadratic grid.
		\( r_i=\frac{R_{\mr{max}}}{\mr{NRMAX}^2} (i+1)^2 \)
	\item[2]
		Logarithmic grid.
		\( r_i=R_{\mr{min}}\mr{exp}
		[\frac{\log(R_{\mr{max}}/R_{\mr{min}})}{\mr{NRMAX}-1}
		i ]\)
	\end{description}
	Since in the case of a point like nucleus, the relativistic
	$s_{1/2}$ and $p_{1/2}$ states are singular at the origin, the
	radial grid never starts with $r$=0.
\item[NVERS]
\item[XCHNG]
\item[NSPIN]
\item[USESIC]
\item[NUCLEUS]
\item[NOSPH]
\item[NOCOLL]
\item[CLEAN\_FIELD]
\item[NONINTOCCNUM]

% Basic parameters
\item[C]
\item[BCONST]
\item[RMIN]
\item[RMAX]
\item[MIXING]
\item[ACCRCY]
\item[ANYSHELL]

% Non-spherical stuff
\item[NLMAX]
\item[EXTRASHELL]
\item[MAXLEXP]
\item[FIXOCCNUM]
\item[NRRECTPT]

% Output control
\item[PRINT\_BASE]
\item[WRITE\_COUPLING]
\item[WRITE\_DELTAW]
\item[NSPHORBSIC]
\item[COMPTON\_PROFILE]
\item[PRINTRHO]
\item[PRINTMAGN]
\item[PRINTPOT]
\item[PRINTENRG]
\item[R\_POWERS]
\item[IPRINT]

\end{description}
\end{document}
